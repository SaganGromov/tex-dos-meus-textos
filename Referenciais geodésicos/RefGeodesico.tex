\input{notationGeo.tex}
\documentclass[a4paper, 12pt, twoside]{article}
\usepackage[utf8]{inputenc}
\usepackage[T1]{fontenc}
\usepackage[portuguese]{babel}
\usepackage[dvipsnames]{xcolor}
\usepackage{amsmath, amsfonts}
\usepackage{amsthm}
\usepackage{etoolbox}
\usepackage{lmodern}
\usepackage{lastpage}
\usepackage{totcount}
\everymath{\displaystyle}
%\usepackage[sc]{mathpazo}
%\linespread{1.05} 
\usepackage{mathrsfs}
\usepackage{faktor}


\definecolor{navybluegalaxy}{RGB}{0, 36, 93}
\newcommand{\bigslant}[2]{{\raisebox{.2em}{$#1$}\left/\raisebox{-.2em}{$#2$}\right.}}

\newcommand{\mycitep}[1]{{\color{teal}\textbf{\cite{#1}}}}
\definecolor{gal}{RGB}{0, 7, 111}






\makeatletter
\patchcmd{\f@nch@head}{\rlap}{\color{BlueViolet}\rlap}{}{}
%\patchcmd{\headrule}{\hrule}{\color{TealBlue}\hrule}{}{}
\patchcmd{\f@nch@foot}{\rlap}{\color{BlueViolet}\rlap}{}{}
\patchcmd{\footrule}{\hrule}{\color{green}\hrule}{}{}
\makeatother

\newtheorem{exerc}{Questão}
\usepackage{float,framed}
\setlength{\intextsep}{2pt}
\setlength{\textfloatsep}{2pt}
\newfloat{Box}{H}{0ob}
\newenvironment{Mybox}{\begin{Box}\begin{framed}\begin{exerc}}{\end{exerc}\end{framed}\end{Box}}



%\usepackage[shortlabels]{enumitem}

\usepackage[a4paper,bottom=0.9in,top=0.9in,left=0.3in,right=0.3in]{geometry}

\usepackage{mathtools}
\usepackage{fancyhdr}
\usepackage{lipsum}
\usepackage{enumerate}
\usepackage{enumitem}

\usepackage{lastpage}
\usepackage{graphicx}
\everymath{\displaystyle}
\newcommand{\p}{\partial}
\pagestyle{fancy}
%\renewcommand{\footrulewidth}{0.4pt}
\fancyhf{}
%\rhead{\color{BlueViolet}\textbf{2022/1}}
%\chead{\textbf{\thepage}}
\lhead{\color{BlueViolet}\textbf{\textit{Referenciais geodésicos}}}
\lfoot{\textbf{MATHEUS A. R. M. HORÁCIO}}
%\rfoot{\textbf{MATRÍCULA: 17/0110923 }}
\rfoot{\textbf{Página \thepage \ de \pageref*{LastPage}}}
  \renewcommand\headrule{%

 \color{BlueViolet}\noindent\makebox[\linewidth]{\rule{\paperwidth}{1pt}}
}
  \renewcommand\footrule{%

 \color{BlueViolet}\noindent\makebox[\linewidth]{\rule{\paperwidth}{1pt}}
}




\newcommand{\om}{\mathbb{M}}

\newcommand{\jps}[1]{\textcolor{blue}{#1}}
%\newcommand{\red}[1]{\textcolor{red}{#1}}
\newcommand{\pur}[1]{\textcolor{purple}{#1}}
\newcommand{\maggg}[1]{\textcolor{magenta}{#1}}

%\usepackage{color}
%\definecolor{SAEblue}{rgb}{0, .62, .91}
%\renewcommand\theequation{\red{{\arabic{equation}}}}


\makeatletter
\let\mytagform@=\tagform@
\def\tagform@#1{\maketag@@@{\bfseries\jps{(\ignorespaces#1\unskip\@@italiccorr)}}\hspace{3mm}}
\renewcommand{\eqref}[1]{\textup{\mytagform@{\ref{#1}}}}
\makeatother


%\chead{\textbf{\thepage}}
\theoremstyle{definition}
\newtheorem{def*}{Definição}
\newtheorem{quest}{Questão}
\newtheorem{quest2}{Questão}
\newcommand{\ve}{\varepsilon}
\newcommand{\lnr}{\left\|}
\newcommand{\ssum}{\displaystyle\sum}
\newcommand{\rnr}{\right\|}
%\newcommand{\R}{\mathbb{R}}
\newcommand{\C}{\mathbb{C}}
\DeclareMathOperator{\hol}{Hol}
\newtheorem*{obs*}{Notação}
%\newtheorem*{oobs}{Observação}
\newtheorem{sublema}{Sub-lema}
\renewcommand{\qedsymbol}{\rule{0.7em}{0.7em}}
\newenvironment{demm}{\smallskip \noindent{\bf \underline{Demonstração:}}}
{\begin{flushright} $\qedsymbol$\end{flushright}\smallskip}


\allowdisplaybreaks

\usepackage{tikz}

\newcommand\PlaceText[3]{%
\begin{tikzpicture}[remember picture,overlay]
\node[outer sep=0pt,inner sep=0pt,anchor=south west] 
  at ([xshift=#1,yshift=-#2]current page.north west) {#3};
\end{tikzpicture}%
}


\newtheoremstyle{theoremDEF}% name of the style to be used
  {\topsep}% measure of space to leave above the theorem. E.g.: 3pt
  {\topsep}% measure of space to leave below the theorem. E.g.: 3pt
  {}% name of font to use in the body of the theorem
  {1pt}% measure of space to indent
  {\bfseries\color{cyan}}% name of head font
  {}% punctuation between head and body
  { }% space after theorem head; " " = normal interword space
  {\underline{\thmname{#1} (\thmnumber{D.#2})\textbf{\thmnote{ (#3)}.}}}

\theoremstyle{theoremDEF}
\newtheorem{deff}{Definição}

\newtheoremstyle{theoremEX}% name of the style to be used
  {\topsep}% measure of space to leave above the theorem. E.g.: 3pt
  {\topsep}% measure of space to leave below the theorem. E.g.: 3pt
  {}% name of font to use in the body of the theorem
  {1pt}% measure of space to indent
  {\bfseries\color{BlueViolet}}% name of head font
  {}% punctuation between head and body
  { }% space after theorem head; " " = normal interword space
  {\underline{\thmname{#1} (\thmnumber{E.#2})\textbf{\thmnote{ (#3)}.}}}

\theoremstyle{theoremEX}
\newtheorem{exem}{Exemplo}

\newtheoremstyle{theoremOOBS}% name of the style to be used
  {\topsep}% measure of space to leave above the theorem. E.g.: 3pt
  {\topsep}% measure of space to leave below the theorem. E.g.: 3pt
  {}% name of font to use in the body of the theorem
  {1pt}% measure of space to indent
  {\bfseries\color{violet}}% name of head font
  {}% punctuation between head and body
  { }% space after theorem head; " " = normal interword space
  {\underline{\thmname{#1} (\thmnumber{O.#2})\textbf{\thmnote{ (#3)}.}}}

\theoremstyle{theoremOOBS}
\newtheorem{oobs}{Observação}


\newtheoremstyle{theoremNOT}% name of the style to be used
  {\topsep}% measure of space to leave above the theorem. E.g.: 3pt
  {\topsep}% measure of space to leave below the theorem. E.g.: 3pt
  {}% name of font to use in the body of the theorem
  {1pt}% measure of space to indent
  {\bfseries\color{cyan}}% name of head font
  {}% punctuation between head and body
  { }% space after theorem head; " " = normal interword space
  {\underline{\thmname{#1} (\thmnumber{N.#2})\textbf{\thmnote{ (#3)}.}}}

\theoremstyle{theoremNOT}
\newtheorem{nott}{Notação}


\newtheoremstyle{theoremLEM}% name of the style to be used
  {\topsep}% measure of space to leave above the theorem. E.g.: 3pt
  {\topsep}% measure of space to leave below the theorem. E.g.: 3pt
  {}% name of font to use in the body of the theorem
  {1pt}% measure of space to indent
  {\bfseries\color{navybluegalaxy}}% name of head font
  {}% punctuation between head and body
  { }% space after theorem head; " " = normal interword space
  {\underline{\thmname{#1} (\thmnumber{L.#2})\textbf{\thmnote{ (#3)}.}}}

\theoremstyle{theoremLEM}
\newtheorem{lema}{Lema}

\newtheoremstyle{theoremTEO}% name of the style to be used
  {\topsep}% measure of space to leave above the theorem. E.g.: 3pt
  {\topsep}% measure of space to leave below the theorem. E.g.: 3pt
  {\itshape}% name of font to use in the body of the theorem
  {5pt}% measure of space to indent
  {\bfseries\color{gal}}% name of head font
  {}% punctuation between head and body
  { }% space after theorem head; " " = normal interword space
  {\underline{\thmname{#1} (\thmnumber{T.#2})\textbf{\thmnote{ (#3)}.}}}
  


\theoremstyle{theoremTEO}
\newtheorem{teorema}{Teorema}



\usepackage{csquotes}
\usepackage{lettrine}
\newcommand{\quotes}[1]{``#1''}
\newcommand{\bl}[1]{\textnormal{\textcolor{black}{#1}}}
\newcommand{\colch}[1]{\left\{ #1 \right\}}



\makeatletter
\let\NAT@parse\undefined
\makeatother


\usepackage{hyperref}
\hypersetup{
	pagebackref=true,
    colorlinks=true, %set true if you want colored links
    linktoc=all,     %set to all if you want both sections and subsections linked
    linkcolor=black,
    citecolor = teal  %choose some color if you want links to stand out
}

\usepackage{cleveref}


\crefname{deff}{}{definitions}
\crefname{exem}{}{exemplos}
\crefname{col}{}{corolários}
\crefname{equation}{}{equações}
\creflabelformat{equation}{#2{\bf{\color{blue}(#1)}}#3}
\crefname{teorema}{}{teoremas}
\crefname{oobs}{observação}{observações}
\creflabelformat{oobs}{#2\bf{\color{violet}(O.#1)}#3}
\crefname{lema}{}{lemas}
\creflabelformat{lema}{#2\bf{\color{gal}(L.#1)}#3}
\crefname{proposicao}{}{proposições}
\creflabelformat{deff}{#2\bf{\color{cyan}(D.#1)}#3}
\creflabelformat{exem}{#2\bf{\color{BlueViolet}(E.#1)}#3}
\creflabelformat{col}{#2\bf{\color{Aquamarine}(C.#1)}#3}
\creflabelformat{proposicao}{#2{\bf\color{Sepia}(P.#1)}#3}
%\renewcommand{\theenumi}{\Alph{enumi}}

\newcommand{\Hess}{\text{Hess}}
\newcommand{\grad}{\text{grad}}
\newcommand{\bb}{\mathcal{B}}
\newcommand{\conjunto}[2]{\{#1 \ \vert \ #2 \}}





\begin{document}

\PlaceText{69mm}{38mm}{ \color{gal}\noindent\makebox[\linewidth]{\rule{2\paperwidth}{1pt}}}

\PlaceText{69mm}{14.3mm}{ \color{white}\noindent\makebox[\linewidth]{\rule{2\paperwidth}{11pt}}}

\PlaceText{69mm}{15mm}{ \color{gal}\noindent\makebox[\linewidth]{\rule{2\paperwidth}{1pt}}}

\PlaceText{69mm}{19mm}{ \color{white}\noindent\makebox[\linewidth]{\rule{2\paperwidth}{3pt}}}

\PlaceText{63mm}{31mm}{\Huge \textcolor{gal}{\textit{Referenciais geodésicos}} }

%\vspace{1cm}

\lettrine[nindent=2em,lines=1]{U}ma das ferramentais mais úteis aos cálculos de geometria Riemanniana é o uso de \emph{referenciais geodésicos}. Duas perguntas que surgem naturalmente aos alunos são as seguintes:
\begin{itemize}
\item como provar a existência de refenciais geodésicos?
\item porque não há perda de generalidade nos cálculos ao usar referenciais geodésicos?
\end{itemize}
O propósito desse texto é responder a primeira pergunta. Para ver uma resposta em detalhes da segunda, consulte \mycitep{MeuTextoLivre}. 

\begin{deff}
Geodésicas partindo de $p$ cujas imagens estão contidas numa vizinhança normal de $p$ são chamadas de \emph{geodésicas radiais}.
\end{deff}
\begin{teorema}\label{refg}
\textit{
Seja $(\mm^n, g)$ uma variedade Riemanniana e $p \in \mm$ um ponto arbitrariamente fixado. Então existe uma vizinhança $U_p \ni p$ e um referencial local $\{E_i\}_{1 \leq i \leq n} \subset \Gamma(TU_p)$ que satisfaz
\[
g(E_i(q), E_j(q)) = \delta_{ij} \ \forall q \in U_p, \text{ e } \nabla_{E_i(p)} E_j = 0,
\]
para quaisquer $1 \leq i, j \leq n$.}
\end{teorema}
\begin{demm}
Seja $U_p$ uma vizinhança normal de $p$ e fixe uma base ortonormal $\{ b_i\}_{1 \leq i \leq n}$ de $T_p \mm$. Essa base induz um isomorfismo
\begin{align*}
B: \mathbb{R}^n &\to T_p \mm \\
(x^1, \cdots, x^n) &\mapsto \sum_{1 \leq i \leq n} x^i \, b_i. 
\end{align*}
Considere agora a carta $\varphi = \left(\exp_p \circ B\right)^{-1}$ em torno de $p$ e seja $\left\{\left.\frac{\partial}{\partial x^i} \right|_{p} \right\}_{1 \leq i \leq n}$ sua base coordenada associada. Uma vez que
\begin{align*}
\left.\frac{\partial}{\partial x^i} \right|_{p} &= \dd(\exp_p \circ B)\left(\left.\frac{\partial}{\partial r^i} \right|_{0} \right) \\
&= \Munderbrace{\dd(\exp_p)_0}{=\operatorname{Id}_{T_p \mm}} \circ \ \Munderbrace{\dd B_0}{=B} \left( \left.\frac{\partial}{\partial r^i} \right|_{0} \right) \\
&= B\left( \left.\frac{\partial}{\partial r^i} \right|_{0}\right) \\
&= b_i,
\end{align*}
segue que $\left\{\left.\frac{\partial}{\partial x^i} \right|_{p} \right\}_{1 \leq i \leq n}$ é uma base ortonormal de $T_p \mm$, de forma que podemos portanto definir $E_i(p) = \left.\frac{\partial}{\partial x^i} \right|_{p}$ para cada $1 \leq i \leq n$. Para estender tal base a um referencial ortonormal local, basta lembrarmos do fato de que o transporte paralelo é uma isometria e realizarmos o transporte paralelo ao longo de geodésicas radiais. Mais precisamente, para cada $q \in U_p$, existe uma única geodésica radial $\gamma_{p, q} : [0, 1] \to \mm$ ligando $p$ e $q$, e podemos então definir
\[
E_i(q) = P_{\gamma_{p, q}, 0, 1}(E_i(p)).
\] 
Em respeito à carta $\varphi$, é claro que a representação em coordenadas da única geodésica $\gamma_v : I \to \mm$ partindo de $p$ com velocidade inicial $v \in T_p \mm$ é dada por $t \mapsto t v$, e portanto a equação das geodésicas se escreve como
\[
\frac{\dd^2}{\dd t^2}  \parent{\gamma_v(t)}^k + \sum_{1 \leq i, j \leq n} \Gamma_{ij}^k \parent{\gamma(t)} \cdot \frac{\dd}{\dd t}  \parent{\gamma_v(t)}^i \cdot  \frac{\dd}{\dd t}  \parent{\gamma_v(t)}^j = 0, \ \forall t \in I \text{ e } \forall 1 \leq k \leq n.
\]
Em particular, temos
\[
\sum_{1 \leq i, j \leq n} \Gamma_{ij}^k (p) \, v^i v^j = 0, \ \forall v = \sum_{1 \leq i \leq n} v^i b_i \in T_p \mm, \text{ seja qual for } 1 \leq k \leq n.
\]
Fazendo $v = \partial_a$ para qualquer $1 \leq a \leq n$ fixado, concluímos que $\Gamma_{aa}^k(p) = 0$ sejam quais forem $1 \leq a, k \leq n$. Finalmente, fazendo $v \in \{\partial_a + \partial_b, \partial_{b} - \partial_a \}$ para quaisquer indíces fixados $1 \leq a, b \leq n$ e subtraindo as equações resultantes, obtemos $\Gamma^k_{ab}(p) = 0$ para quaiquer $1 \leq a, b, k \leq n$, donde segue que $(\nabla_{E_i} E_j)(p) = 0$ sejam quais forem $1 \leq i, j \leq n$.
\end{demm}
\begin{oobs}
É fácil verificar que em coordenadas normais, todas as primeiras derivadas parciais de $g_{ij}$ se anulam em $p$. Isso mostra que não existem invariantes geométricos de ordem $< 2$. Como visto em \mycitep{leeriem}, tal fato também pode ser visto pela expansão da métrica em coordenadas normais, dada por
\[
g_{ij}(t) = \delta_{ij} + \frac{1}{3} \, \sum_{1 \leq k, \ell \leq n}  \Rm_{ik \ell j} \, x^k \, x^{\ell} + O\parent{|t|^3}.
\]
\end{oobs}
\begin{oobs}\label{XrefG}
Pela tensorialidade de $\nabla$ na primeira entrada, é claro também que o referencial geodésico do lema \cref{refg} satisfaz
\[
\nabla_{v} E_i = 0,
\]
seja qual for $v \in T_p \mm$.
\end{oobs}

\begin{thebibliography}{9}

{\bfseries
\color{teal}






\bibitem{MeuTextoLivre}
\bl{
\textbf{Horácio, M.A.R.M.} \emph{One chart to rule them all!} Texto encontrado em página pessoal. Disponível em: \href{https://sagangromov.github.io/assets/pdf/OneChart.pdf}{\textbf{https://sagangromov.github.io/assets/pdf/OneChart.pdf}}. Acessado em 15 de fevereiro de 2023.
}


\bibitem{Dissertacao}
\bl{\textbf{Horácio, M.A.R.M.} Estimativas de curvatura para sólitons de Ricci gradiente quadridimensionais. Dissertação de Mestrado, Universidade de Brasília, (2023).}


\bibitem{leesmooth}
\bl{
\textbf{Lee, John M.} Introduction to smooth manifolds. Second edition. Graduate Texts in Mathematics, 218. Springer, New York, 2013. xvi+708 pp. ISBN: 978-1-4419-9981-8.
}

\bibitem{leeriem}
\bl{
\textbf{Lee, John M.} Introduction to Riemannian manifolds. Second edition of [MR1468735]. Graduate Texts in Mathematics, 176. Springer, Cham, 2018. xiii+437 pp. ISBN: 978-3-319-91754-2; 978-3-319-91755-9.
}




}




\end{thebibliography}

\end{document}
